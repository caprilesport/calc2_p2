\documentclass[a4paper, 12pt]{article}

\usepackage[portuges]{babel}
\usepackage[utf8]{inputenc}
\usepackage{amsmath}
\usepackage{indentfirst}
\usepackage{graphicx}
\usepackage{multicol,lipsum}

\begin{document}
%\maketitle

\begin{titlepage}
	\begin{center}

		%\begin{figure}[!ht]
		%\centering
		%\includegraphics[width=2cm]{c:/ufba.jpg}
		%\end{figure}

		\Huge{Universidade Federal de Santa Catarina}\\
		\large{Departamento de Matemática}\\
		\vspace{15pt}
		\vspace{95pt}
		\textbf{\LARGE{Prova 2 e Lista de exercícios - H-Calculo II }}\\
		%\title{{\large{Título}}}
		\vspace{3,5cm}
	\end{center}

	\begin{flushleft}
		\begin{tabbing}
			Aluno:  Vinícius Capriles Port\\
			Professor: Marianna Ravara Vago\\
		\end{tabbing}
	\end{flushleft}
	\vspace{1cm}

	\begin{center}
		\vspace{\fill}
		Dezembro\\
		2022
	\end{center}
\end{titlepage}
%%%%%%%%%%%%%%%%%%%%%%%%%%%%%%%%%%%%%%%%%%%%%%%%%%%%%%%%%%%

% % % % % % % % %FOLHA DE ROSTO % % % % % % % % % %

% % % % % % % % % % % % % % % % % % % % % % % % % %
\newpage
\tableofcontents
\thispagestyle{empty}

\newpage
\pagenumbering{arabic}
% % % % % % % % % % % % % % % % % % % % % % % % % % %
\section{Aproximação por funções polinomiais}

Embora tenhamos um conjunto de funções elementares, muitas delas são difíceis de calcular,
como, por exemplo, $e^x$, $log(x)$, $sen(x)$ e $cos(x)$. Em contraste,
as funções polinomiais são muito mais simples de serem computadas.
Assim, se houver uma forma de aproximar uma função qualquer $f(x)$ por uma função polinomial,
poderemos então calcular os valores desta função para uma gama muito maior de valores.

O polinômio a seguir:

\begin{centering}
	\begin{equation}
		P_{n,a}(x) = a_0 + a_1(x-a) + a_2(x-a) + ... + a_n(x-a)
	\end{equation}
\end{centering}

é chamado de polinômio de Taylor de grau $n$ para $f$ em $a$,
onde $a_k = \frac{f^{(k)}(a)}{k!}$, com $0 \leq k \leq n$.

Apesar da obtenção do Polinômio de Taylor parecer complicada, para funções como
$e^x$, $sin(x)$ e $cos(x)$ é surpreendentemente simples obtê-lo, já que suas derivadas são facilmente obtidas.

Por exemplo, uma vez que a função $e^x$ possui como k-ésima derivada $f^{k} = e^x$, seu
polinômio de Taylor de grau n centrado em 0 é:

\begin{centering}
	\begin{equation}
		P_{n,a,e^x}(x) = 1 + \frac{x}{1!} + \frac{x^2}{2!} + \frac{x^3}{3!} + ...  + \frac{x^n}{n!}
	\end{equation}
\end{centering}

Já para função logaritmica, uma vez que esta não está definida para $x = 0$, pode-se obter o
polinômio de Taylor para a função $f(x) = log(1+x)$, para então se obter um polinômio que
está novamente centrado em $0$ e portanto tem como elements $(x-a)^n = x^n$.

Embora ainda não saibamos qual a exata conexão entre o polinômio de Taylor de uma função $f(x)$,
já podemos perceber que ele é muito util para se obter uma aproximação para $f(x)$.

Ao se analisar o polinômio de Taylor de grau 1 para uma função qualquer,

\begin{centering}
	\begin{align}
		P_{1,a}(x) = f(a) + f'(a)(x-a) \\
		\frac{f(x) - P_{1,a}(x)}{x-a} = \frac{f(x) - f(a)}{x-a} - f'(a)
	\end{align}
\end{centering}

Como a razão $\frac{f(x) - f(a)}{x-a}$ é a propria definição de derivada, temos que a primeira
razão é igual a $0$, o que indica que $P_{1,a}$ se aproxima de $f(x)$ mais rápido do que $x$
se aproxima de $a$. Este resultado pode ser extendido para um polinômio de grau $n$, que é
apresentado no teorema 1 deste capitulo, que diz que

\begin{equation}
	\lim_{x \to a } \frac{f(x) - P_{n,a}(x)}{(x-a)^n}
\end{equation}

e assim, começamos a revelar a relação que um polinômio desta forma tem com uma
função genérica.

Uma das consequencias do teorema 1 é fornecer uma maneira de se analisar os pontos de
máximo e minimo para uma função. Como ja sabiamos, se $a$ é um ponto critico e $f''(a)$ existe,
e é $ < 0$, temos que este é um ponto de mínimo local, já quando $f''(a)$ é $ > 0$ se trata de um ponto
de máximo local. Agora, caso $f''(a) = 0$, podemos começar a analisar $f'''(a)$, ou,
de uma maneira mais abrangente, tentar entender o que ocorre quando

\begin{equation}
	f''(a) = f'''(a) = .... = f^{(n-1)} = 0
\end{equation}

e

\begin{equation}
	f^{(n)} \neq 0
\end{equation}

Como todas as derivadas de ordem $ < n $ são $=0$, temos que o polinômio de taylor para
esta função será

\begin{equation}
	P_{n,a} = a_n(x-a) = \frac{f^{(n)}(a)(x-a)^n}{n!}
\end{equation}

Ou seja, basta analisar a paridade de $n$ e o sinal de $f^{(n)}(a)$, já que $f^{(n)}(a)$
é uma constante e $(x-a)^n$ pode ser analisado como qualquer polinômio, tendo seu
formato geral já conhecido. Sendo assim:

\begin{itemize}
	\item se $n$ é par e $f^{(n)}(a)$ $>0$, então $a$ é um mínimo local.
	\item se $n$ é par e $f^{(n)}(a)$ e $<0$, então $a$ é um máximo local.
	\item se $n$ é impar, então $a$ não é nem máximo, nem mínimo.
\end{itemize}

Assim, este desenvolvimento dos Polinômios de Taylor conseguiu complementar o teste das
derivadas para obter informações de uma função em seus pontos criticos. Embora esta
ferramente dê conta de diversas funções, caso uma função tenha $f^{(n)}(a) = 0$ $ \forall n$
não ganhamos nenhum poder de conclusão neste momento.

Através deste teorema, também podemos introduzir o conceito de ordem de igualdade,
ou seja, o que são duas funções \textit{iguais até ordem n}.

Duas funções são iguais até ordem $n$ se

\begin{equation}
	\lim_{x \to a }\frac{f(x) - g(x)}{(x-a)^n} = 0
\end{equation}

Utilizando este conceito, podemos agora apresentar o teorema 3:

O teorema 3 diz que se $P$ e $Q$ são dois polinômios em (x-a) de grau $\leq n$ e $P$
e $Q$ são iguais até ordem $n$, então $P=Q$.

Aplicando este teorema, podemos concluir que se $f(x)$ é $n$ vezes
diferenciável e $P$ é um polinômio de ordem $\leq n$ em $(x-a)$, então $P = P_{n,a}$, mostrando
que este é o unico polinômio que é igual até ordem n em relação a uma função $f(x)$. Ou seja,
mostramos que o polinômio de Taylor é o unico
polinômio de grau $\leq n $ que tem a propriedade de aproximar tão bem uma determinada função.

Através deste teorema, o Spivak constroi a noção de resto aplicando ele a função
$arctan(x)$. Este resto é denotado por $R_{n,a}(x)$ e:

\begin{equation}
	f(x) = P_{n,a} + R_{n,a}
\end{equation}

O que faz sentido, jã que se estamos aproximando $f(x)$ por um polinômio, e quanto mais aumentamos
o grau deste mais próximo chegamos de $f(x)$, deve haver uma maneira de expressar o erro
que obtemos ao utilizar esta aproximação. Uma das maneiras de se expressar este resto é considerando
primeiramente o que ocorre para o polinômio de grau $0$,

\begin{align}
	f(x) = P_{0,a} + R_{0,a} \\
	f(x) = f(a) + R_{0,a}    \\
	R_{0,a} = f(x) - f(a)
\end{align}

Ou seja:

\begin{equation}
	R_{0,a} = \int_{a}^{x} f'(y)dy
\end{equation}

Extrapolando por indução obtemos:

\begin{equation}
	R_{n,a} = \int_{a}^{x} \frac{f^{(n+1)}(y)}{n!}(x-y)^n dy
\end{equation}

E agora temos uma maneira de calcular o resto para uma função $f(x)$ qualquer. Ou seja,
podemos calcular um valor para $f(b)$ com uma precisão desejada, bastando calcular o resto para que
este seja menor do que a precisão desejada, e assim, sabendo qual o grau necessário para o
polinômio.

Outras expressões para o resto surgem a partir do teorema de Taylor:

Supondo que $f',...,f^{(n+1)}$ estão definidas em $[a,x]$ e que $R_{n,a}(x)$ está definido por:

\begin{equation}
	f(x) = f(a) + f'(a)(x-a) + .... + \frac{f^{(n)}(a)}{n!} + R_{n,a}
\end{equation}

Então:

\begin{itemize}
	\item $R_{n,a}(x) = \frac{f^{(n+1)}(t)}{n!}(x-t)^n (x-a)$
	\item $R_{n,a}(x) = \frac{f^{(n+1)}(t)}{(n + 1)!}(x-a)^{n+1}$
\end{itemize}

Considerando que $\frac{x^n}{n!} \leq \epsilon$ para algum n suficientemente
grande podemos, portanto, calcular o valor de $f(x)$ com a precisão desejada. 

Isto é muito interessante devido a aplicação direta deste resultado, com uma maneira
de se calcular $R_{n,a}(x)$, pode-se aplicar isto a métodos numéricos e então
implementar o cálculo de diversas funções com a precisão desejada, o que com certeza
é muito aplicado em calculadores, linguagens de programação entre outros domnínios
para se ter um tipo de rotina que obtem valores precisos para $log(x)$, $e^x$ entre
outras funções comumente encontradas em diferentes domínios científicos.


\newpage

\section{Sequências Infinitas}

Agora que temos uma maneira de gerar sequências infinitas (uma vez que elas surgem
naturalmente a partir dos polinômios de Taylor) podemos então nos aprofundar nos
estudos deste tipo de objeto.

Inicialmente, o Spivak define uma sequência inifita de números reais como sendo uma função
cujo dominio é $\mathbf{N}$. Em seguida, define também o que significa dizer que uma sequência
infinita converge.

A definição de convergência para sequências infinitas é análoga a definição fornecida
para um limite com ${x \to \infty}$:

``Uma sequência \{$a_n$\} \textbf{converge para $l$} ($ \lim_{n \to \infty} a_n = l $) ''  se para cada
$\epsilon > 0$ existe um número natural $N$ tal que, $\forall n$ se , 

\begin{equation}
	n > N \to |a_n - l| < \epsilon 
\end{equation}

onde uma sequência converge se ela se aproxima de $l$ para algum $l$ qualquer, e diverge
caso contrário.

Após estas definições e alguns exemplos de como analisar se algumas sequências convergem
(como a sequência $\{\sqrt{n+1} - \sqrt{n}\}$), bem como a análise de algumas propriedades 
de soma, multiplicação e divisão de sequências
que assim como em limites que $\to \infty$ são verdadeiras, devido a similaridade entre
estas sequências e limites tendendo a infinito.

Esta similaridade não é mero acaso (teorema 2), existe uma conexão entre uma funcão $f(x)$ que
satisfaz

\begin{equation}
	\lim_{x \to c} f(x) = l
\end{equation}

e uma sequência convergente estar contida dentro do domínio de $f(x)$ tal que
	
\begin{equation}
	\lim_{n \to \infty} f(a_n) = l
\end{equation}

que, honestamente, me parece meio redundante, uma vez que já preciso ter a informação
da convergência de $\{a_n\}$ para usar deste teorema. Apesar disso, parece que isso facilita a 
obtenção de novas séries, uma vez que podemos aplicar uma série convergente a diversas
funções e com isso obter novas séries.

Após a explicação deste teorema, o Spivak passa a definir um critério que é de fundamental 
importância daqui pra frente, e que também apresenta uma analogia com conceitos de 
funções já estudadas anteriormente na matéria. 

Assim como para funções definimos os conceitas de função crescente (se $a_{n+1} > a_n \forall n$),
não-decresente (se $a_{n+1} \geq a_n \forall n$) e limitada por cima (se existe um $M$ t.q.
$a_n \leq M \forall n$), podemos também definir estes conceitas para sequências.

Uma das grandes utilidades destes conceitas se materializa no teorema 2, que diz que:

\begin{itemize}
	\item Se $\{a_n\}$ é não-decrescente
	\item Se $\{a_n\}$ é limitada por cima
\end{itemize}

então $\{a_n\}$ converge.

Como comentado pelo Spivak, apesar de que este teorema parece de certa forma pouco poderoso,
uma vez que necessita da hipótese de que $\{a_n\}$ é limitada e não-decrescente, se
aplicando a poucas sequências, ele é de uma grande utilidade após se tomar conhecimento
do lema apresentado. Este lema diz que é sempre possível extrair uma subsequência de 
uma sequência t.q. a subsequência seja não-decrescente ou não-crescente.

Após estas definições, é definido o que é uma sequência de Cauchy:

Se para cada $\epsilon > 0$ existe um numero natural $N$ t.q. $\forall m,n$:

\begin{equation}
	se\  m,n > N \to |a_n - a_m| < \epsilon
\end{equation}

Onde esta definição nos leva ao teorema 3, que prova que se uma sequência $\{a_n\}$ é uma 
sequência de Cauchy, então esta sequência é convergente.

Assim, ao final deste capítulo foram introduzidos alguns conceitos importantes acerca 
de sequências, que faz que com que agora tenhamos a base para trabalhar com séries,
ou seja, a soma destas consequências.

\newpage


\section{Séries Infinitas}






\newpage
\section{Convergência uniforme e séries de potência}

\newpage

\section{Exercicios Cap. 20}

1 - IV

10 - (a)


(b)

13

\newpage
\section{Exercicios Cap. 22}

1 (iv), 4, 20

\newpage

\section{Exercicios Cap. 23}

1 (ii, iv, viii, xx), 15

\newpage

\section{Exercicios Cap. 24}

4, 6, 14

\end{document}
