\documentclass[a4paper, 12pt]{article}

\usepackage[portuges]{babel}
\usepackage[utf8]{inputenc}
\usepackage{amsmath}
\usepackage{indentfirst}
\usepackage{graphicx}
\usepackage{multicol,lipsum}

\begin{document}
%\maketitle

\begin{titlepage}
	\begin{center}

		%\begin{figure}[!ht]
		%\centering
		%\includegraphics[width=2cm]{c:/ufba.jpg}
		%\end{figure}

		\Huge{Universidade Federal de Santa Catarina}\\
		\large{Departamento de Matemática}\\
		\vspace{15pt}
		\vspace{95pt}
		\textbf{\LARGE{Prova 2 - H-Calculo II }}\\
		%\title{{\large{Título}}}
		\vspace{3,5cm}
	\end{center}

	\begin{flushleft}
		\begin{tabbing}
			Aluno:  Vinícius Capriles Port\\
			Professora: Marianna Ravara Vago\\
		\end{tabbing}
	\end{flushleft}
	\vspace{1cm}

	\begin{center}
		\vspace{\fill}
		Dezembro\\
		2022
	\end{center}
\end{titlepage}
%%%%%%%%%%%%%%%%%%%%%%%%%%%%%%%%%%%%%%%%%%%%%%%%%%%%%%%%%%%

% % % % % % % % %FOLHA DE ROSTO % % % % % % % % % %

% % % % % % % % % % % % % % % % % % % % % % % % % %
\newpage
\tableofcontents
\thispagestyle{empty}

\newpage
\pagenumbering{arabic}
% % % % % % % % % % % % % % % % % % % % % % % % % % %
\section{Aproximação por funções polinomiais}

Embora tenhamos um conjunto de funções elementares, muitas delas são difíceis de calcular,
como, por exemplo, $e^x$, $log(x)$, $sen(x)$ e $cos(x)$. Em contraste,
as funções polinomiais são muito mais simples de serem computadas.
Assim, se houver uma forma de aproximar uma função qualquer $f(x)$ por uma função polinomial,
poderemos então calcular os valores desta função para uma gama muito maior de valores.

O polinômio a seguir:

\begin{centering}
	\begin{equation}
		\nonumber P_{n,a}(x) = a_0 + a_1(x-a) + a_2(x-a) + ... + a_n(x-a)
	\end{equation}
\end{centering}

é chamado de polinômio de Taylor de grau $n$ para $f$ em $a$,
onde $a_k = \frac{f^{(k)}(a)}{k!}$, com $0 \leq k \leq n$.

Apesar da obtenção do Polinômio de Taylor parecer complicada, para funções como
$e^x$, $sin(x)$ e $cos(x)$ é surpreendentemente simples obtê-lo, já que suas derivadas são facilmente obtidas.

Por exemplo, uma vez que a função $e^x$ possui como k-ésima derivada $f^{k} = e^x$, seu
polinômio de Taylor de grau n centrado em 0 é:

\begin{centering}
	\begin{equation}
		\nonumber P_{n,a,e^x}(x) = 1 + \frac{x}{1!} + \frac{x^2}{2!} + \frac{x^3}{3!} + ...  + \frac{x^n}{n!}
	\end{equation}
\end{centering}

Já para função logaritmica, uma vez que esta não está definida para $x = 0$, pode-se obter o
polinômio de Taylor para a função $f(x) = log(1+x)$, para então se obter um polinômio que
está novamente centrado em $0$ e portanto tem como elements $(x-a)^n = x^n$.

Embora ainda não saibamos qual a exata conexão entre o polinômio de Taylor de uma função $f(x)$,
já podemos perceber que ele é muito util para se obter uma aproximação para $f(x)$.

Ao se analisar o polinômio de Taylor de grau 1 para uma função qualquer,

\begin{centering}
	\begin{align*}
		P_{1,a}(x) = f(a) + f'(a)(x-a) \\
		\frac{f(x) - P_{1,a}(x)}{x-a} = \frac{f(x) - f(a)}{x-a} - f'(a)
	\end{align*}
\end{centering}

Como a razão $\frac{f(x) - f(a)}{x-a}$ é a propria definição de derivada, temos que a primeira
razão é igual a $0$, o que indica que $P_{1,a}$ se aproxima de $f(x)$ mais rápido do que $x$
se aproxima de $a$. Este resultado pode ser extendido para um polinômio de grau $n$, que é
apresentado no teorema 1 deste capitulo, que diz que

\begin{equation}
	\nonumber \lim_{x \to a } \frac{f(x) - P_{n,a}(x)}{(x-a)^n}
\end{equation}

e assim, começamos a revelar a relação que um polinômio desta forma tem com uma
função genérica.

Uma das consequencias do teorema 1 é fornecer uma maneira de se analisar os pontos de
máximo e minimo para uma função. Como ja sabiamos, se $a$ é um ponto critico e $f''(a)$ existe,
e é $ < 0$, temos que este é um ponto de mínimo local, já quando $f''(a)$ é $ > 0$ se trata de um ponto
de máximo local. Agora, caso $f''(a) = 0$, podemos começar a analisar $f'''(a)$, ou,
de uma maneira mais abrangente, tentar entender o que ocorre quando

\begin{equation}
	\nonumber f''(a) = f'''(a) = .... = f^{(n-1)} = 0
\end{equation}

e

\begin{equation}
	\nonumber f^{(n)} \neq 0
\end{equation}

Como todas as derivadas de ordem $ < n $ são $=0$, temos que o polinômio de taylor para
esta função será

\begin{equation}
	\nonumber P_{n,a} = a_n(x-a) = \frac{f^{(n)}(a)(x-a)^n}{n!}
\end{equation}

Ou seja, basta analisar a paridade de $n$ e o sinal de $f^{(n)}(a)$, já que $f^{(n)}(a)$
é uma constante e $(x-a)^n$ pode ser analisado como qualquer polinômio, tendo seu
formato geral já conhecido. Sendo assim:

\begin{itemize}
	\item se $n$ é par e $f^{(n)}(a)$ $>0$, então $a$ é um mínimo local.
	\item se $n$ é par e $f^{(n)}(a)$ e $<0$, então $a$ é um máximo local.
	\item se $n$ é impar, então $a$ não é nem máximo, nem mínimo.
\end{itemize}

Assim, este desenvolvimento dos Polinômios de Taylor conseguiu complementar o teste das
derivadas para obter informações de uma função em seus pontos criticos. Embora esta
ferramente dê conta de diversas funções, caso uma função tenha $f^{(n)}(a) = 0$ $ \forall n$
não ganhamos nenhum poder de conclusão neste momento.

Através deste teorema, também podemos introduzir o conceito de ordem de igualdade,
ou seja, o que são duas funções \textit{iguais até ordem n}.

Duas funções são iguais até ordem $n$ se

\begin{equation}
	\nonumber \lim_{x \to a }\frac{f(x) - g(x)}{(x-a)^n} = 0
\end{equation}

Utilizando este conceito, podemos agora apresentar o teorema 3:

O teorema 3 diz que se $P$ e $Q$ são dois polinômios em (x-a) de grau $\leq n$ e $P$
e $Q$ são iguais até ordem $n$, então $P=Q$.

Aplicando este teorema, podemos concluir que se $f(x)$ é $n$ vezes
diferenciável e $P$ é um polinômio de ordem $\leq n$ em $(x-a)$, então $P = P_{n,a}$, mostrando
que este é o unico polinômio que é igual até ordem n em relação a uma função $f(x)$. Ou seja,
mostramos que o polinômio de Taylor é o unico
polinômio de grau $\leq n $ que tem a propriedade de aproximar tão bem uma determinada função.

Através deste teorema, o Spivak constroi a noção de resto aplicando ele a função
$arctan(x)$. Este resto é denotado por $R_{n,a}(x)$ e:

\begin{equation}
	\nonumber f(x) = P_{n,a} + R_{n,a}
\end{equation}

O que faz sentido, jã que se estamos aproximando $f(x)$ por um polinômio, e quanto mais aumentamos
o grau deste mais próximo chegamos de $f(x)$, deve haver uma maneira de expressar o erro
que obtemos ao utilizar esta aproximação. Uma das maneiras de se expressar este resto é considerando
primeiramente o que ocorre para o polinômio de grau $0$,

\begin{align*}
	f(x) = P_{0,a} + R_{0,a} \\
	f(x) = f(a) + R_{0,a}    \\
	R_{0,a} = f(x) - f(a)
\end{align*}

Ou seja:

\begin{equation}
	\nonumber R_{0,a} = \int_{a}^{x} f'(y)dy
\end{equation}

Extrapolando por indução obtemos:

\begin{equation}
	\nonumber R_{n,a} = \int_{a}^{x} \frac{f^{(n+1)}(y)}{n!}(x-y)^n dy
\end{equation}

E agora temos uma maneira de calcular o resto para uma função $f(x)$ qualquer. Ou seja,
podemos calcular um valor para $f(b)$ com uma precisão desejada, bastando calcular o resto para que
este seja menor do que a precisão desejada, e assim, sabendo qual o grau necessário para o
polinômio.

Outras expressões para o resto surgem a partir do teorema de Taylor:

Supondo que $f',...,f^{(n+1)}$ estão definidas em $[a,x]$ e que $R_{n,a}(x)$ está definido por:

\begin{equation}
	\nonumber f(x) = f(a) + f'(a)(x-a) + .... + \frac{f^{(n)}(a)}{n!} + R_{n,a}
\end{equation}

Então:

\begin{itemize}
	\item $R_{n,a}(x) = \frac{f^{(n+1)}(t)}{n!}(x-t)^n (x-a)$
	\item $R_{n,a}(x) = \frac{f^{(n+1)}(t)}{(n + 1)!}(x-a)^{n+1}$
\end{itemize}

Considerando que $\frac{x^n}{n!} \leq \epsilon$ para algum n suficientemente
grande podemos, portanto, calcular o valor de $f(x)$ com a precisão desejada. 

Isto é muito interessante devido a aplicação direta deste resultado, com uma maneira
de se calcular $R_{n,a}(x)$, pode-se aplicar isto a métodos numéricos e então
implementar o cálculo de diversas funções com a precisão desejada, o que com certeza
é muito aplicado em calculadores, linguagens de programação entre outros domnínios
para se ter um tipo de rotina que obtem valores precisos para $log(x)$, $e^x$ entre
outras funções comumente encontradas em diferentes domínios científicos.


\newpage

\section{Sequências Infinitas}

Agora que temos uma maneira de gerar sequências infinitas (uma vez que elas surgem
naturalmente a partir dos polinômios de Taylor) podemos então nos aprofundar nos
estudos deste tipo de objeto.

Inicialmente, o Spivak define uma sequência inifita de números reais como sendo uma função
cujo dominio é $\mathbf{N}$. Em seguida, define também o que significa dizer que uma sequência
infinita converge.

A definição de convergência para sequências infinitas é análoga a definição fornecida
para um limite com ${x \to \infty}$:

``Uma sequência \{$a_n$\} \textbf{converge para $l$} ($ \lim_{n \to \infty} a_n = l $) ''  se para cada
$\epsilon > 0$ existe um número natural $N$ tal que, $\forall n$ se , 

\begin{equation}
	\nonumber n > N \to |a_n - l| < \epsilon 
\end{equation}

onde uma sequência converge se ela se aproxima de $l$ para algum $l$ qualquer, e diverge
caso contrário.

Após estas definições e alguns exemplos de como analisar se algumas sequências convergem
(como a sequência $\{\sqrt{n+1} - \sqrt{n}\}$), bem como a análise de algumas propriedades 
de soma, multiplicação e divisão de sequências
que assim como em limites que $\to \infty$ são verdadeiras, devido a similaridade entre
estas sequências e limites tendendo a infinito.

Esta similaridade não é mero acaso (teorema 2), existe uma conexão entre uma funcão $f(x)$ que
satisfaz

\begin{equation}
	\nonumber \lim_{x \to c} f(x) = l
\end{equation}

e uma sequência convergente estar contida dentro do domínio de $f(x)$ tal que
	
\begin{equation}
	\nonumber \lim_{n \to \infty} f(a_n) = l
\end{equation}

que, honestamente, me parece meio redundante, uma vez que já preciso ter a informação
da convergência de $\{a_n\}$ para usar deste teorema. Apesar disso, parece que isso facilita a 
obtenção de novas séries, uma vez que podemos aplicar uma série convergente a diversas
funções e com isso obter novas séries.

Após a explicação deste teorema, o Spivak passa a definir um critério que é de fundamental 
importância daqui pra frente, e que também apresenta uma analogia com conceitos de 
funções já estudadas anteriormente na matéria. 

Assim como para funções definimos os conceitas de função crescente (se $a_{n+1} > a_n \forall n$),
não-decresente (se $a_{n+1} \geq a_n \forall n$) e limitada por cima (se existe um $M$ t.q.
$a_n \leq M \forall n$), podemos também definir estes conceitas para sequências.

Uma das grandes utilidades destes conceitas se materializa no teorema 2, que diz que:

\begin{itemize}
	\item Se $\{a_n\}$ é não-decrescente
	\item Se $\{a_n\}$ é limitada por cima
\end{itemize}

então $\{a_n\}$ converge.

Como comentado pelo Spivak, apesar de que este teorema parece de certa forma pouco poderoso,
uma vez que necessita da hipótese de que $\{a_n\}$ é limitada e não-decrescente, se
aplicando a poucas sequências, ele é de uma grande utilidade após se tomar conhecimento
do lema apresentado. Este lema diz que é sempre possível extrair uma subsequência de 
uma sequência t.q. a subsequência seja não-decrescente ou não-crescente.

Após estas definições, é definido o que é uma sequência de Cauchy:

Se para cada $\epsilon > 0$ existe um numero natural $N$ t.q. $\forall m,n$:

\begin{equation}
	\nonumber se\  m,n > N \to |a_n - a_m| < \epsilon
\end{equation}

Onde esta definição nos leva ao teorema 3, que prova que se uma sequência $\{a_n\}$ é uma 
sequência de Cauchy, então esta sequência é convergente.

Assim, ao final deste capítulo foram introduzidos alguns conceitos importantes acerca 
de sequências, que faz que com que agora tenhamos a base para trabalhar com séries,
ou seja, a soma destas consequências.

\newpage


\section{Séries Infinitas}

No inicio do capítulo, iniciamos definindo o que é uma série infinita, sendo que toda a base
que necessitavamos para definir isso foi obtida no capítulo passado. Olhando para as sequências
anteriormente definidas, a definição de um série infinita vem naturalmente, uma vez que
somando todos os elementos de uma sequência, chegariamos a uma série.

É claro, como nem toda sequência era convergente, também temos problemas com as séries, sendo que
nem toda série é \textbf{somável}. Como por exemplo, a série definida por:

\begin{equation}
	\nonumber a_n = (-1)^{n+1}
\end{equation}

Tendo como série (análisando as somas parciais):

\begin{align*}
	s_1 = a_1 = 1\\
	s_2 = a_1 + a_2 = 0\\
	s_3 = a_1 + a_2 + a_3 = 1\\
	...
\end{align*}

Para computarmos a soma de uma série, faz sentido olharmos o que ocorre para a soma parcial
$\{s_n\}$, assim como fizemos para o exemplo acimadefinida da seguinte maneira:

\begin{equation}
	\nonumber s_n = \sum_{i=1}^{n} a_i
\end{equation}

Onde a nossa série converge e é chamada de \textbf{somável} se a sequência $\{s_n\}$ converge.
Ou seja, se $\lim_{n \to \infty} s_n $ está definido, sendo este valor chamado de soma de sequência.

Após esta definição, começamos pelo critério de Cauchy, que como comentado pelo Spivak
náo é muito útil para demonstrações, entretanto este é de fundamental importância teórica.

O critério de Cauchy diz que, uma sequência $\{a_n\}$ é somável se e somente se,

\begin{equation}
	\nonumber \lim_{m,n \to \infty} a_{n+1} + ... + a_m = 0
\end{equation}

ou seja, ele diz que para que uma sequência seja somável, ela deve se aproximar de $0$
quanto mais $ n\to \infty$.

Como dito, este é um critério mais teórico, sendo que a primeira condição prática para 
que ele ocorra é dado pela condição de desaparecimento (vanishing condition). Que diz
se $\{a_n\}$ é somável, então 


\begin{equation}
	\nonumber \lim_{n \to \infty } a_n = 0.
\end{equation}

Apesar de ser um bom critério inicial, algumas sequencias como a sequencia

\begin{equation}
	\nonumber s_n = 1 + 1/2 + 1/3 + 1/4 + .... + 1/n
\end{equation}


falham frente a ele. O que parece curioso numa primeira inspeção, já que 
\begin{equation}
	\nonumber \lim_{n \to \infty } 1/n = 0
\end{equation}

embora exista uma maneira de se somar esta sequência (agrupando esta em termos que somam $> 1/2$)
fazendo com que ela não seja somável. 

Sendo assim, faz-se necessário introduzir outros critérios para que se 
avalie a convergência de séries. Ao longo do restante do resumo deste capítulo, estes
critérios serão elencados. O primeiro deles é chamado de critério
de delimitação (boundedness criterion). Este critério diz que uma sequência não negativa (isto é,
$a_n \geq 0 $, que faz com que $s_n$ seja claramente não-decrescente) é somável se e somente se
suas somas parciais $s_n$ são limitadas.

Embora este critério não ajude muito no sentido de que de certa forma continuamos no escuro
(ainda não sabemos avaliar se uma série é ou não limitada), se temos alguma outra série auxiliar 
podemos então avaliar outras séries. Isto é evidenciado pelo teorema 1, que diz que:

Se $0 \leq a_n \leq b_n\ \forall \ n$ então se


\begin{equation}
	\nonumber \sum_{n=1}^{\infty}b_n\  \text{converge então } \sum_{n=1}^{\infty} a_n \text{ converge} 
\end{equation}

Este critério faz com que diversas expressões complicadas possam ser avaliadas mais facilmente,
uma vez que funções como seno e cosseno com $n \to \infty$ ainda são limitadas ao intervalo $ -1 \leq x \leq 1$ 
e portanto são sempre $\leq$ a uma série análoga sem a presença destas funções.

Um outro teste similar a este, no sentido de que a partir de uma série convergente pode-se 
obter outras séries também convergentes, é o teste da comparação do limite (limit comparison teste)

Este teste, enunciado como teorema 2, diz que se $a_n,b_n > 0$ e $\lim_{n \to \infty} a_n/b_n = c$, 
então $\sum_{n=1}^{\infty}a_n$ converge se e somente se $\sum_{n=1}^{\infty}b_n$ converge.

Este teorema implica em um dos testes mais importantes para se avaliar a convergência de sérioes, que
é o teste da razão:

Seja $a_n > 0 \ \forall \ n$, e suponha que 

\begin{equation}
 \nonumber \lim_{n\to\infty} \frac{a_{n+1}}{a_n} = r
\end{equation}

então, se $r < 1$ a série é convergente, se $r>1$ a série é divergente e se $r=1$ o teste
é inconclusivo.

Este teste é muito interessante, pois se for possível avaliar $ \lim_{n\to\infty} \frac{a_{n+1}}{a_n} $
temos um critério que depende apenas da proprio comportamento da série para se avaliar sua 
convergência (assim como o critério de Cauchy, apesar de este parecer ser mais poderoso). Entretanto, muitas vezes é dificil de se avaliar $ \lim_{n\to\infty} \frac{a_{n+1}}{a_n} $,
bem como este pode ser $=1$, o que é inconclusivo (a série pode ser tanto convergente como divergente).

Embora estes critérios todos tenham limitações, pode-se perceber que já começamos a criar um
arsenal de critérios para se avaliar séries, podendo se escolher um a depender do que é mais fácil de se
avaliar. O próximo critério apresentado (o teste da integral, teorema 4) adiciona mais um 
critério a este arsenal.

O teorema 4 supõe uma função $f$ positiva e decrescente em $[1,\infty)$, sendo $f(n) = a_n \ \forall \ n$.
Então, $\sum a_n$ converge se e somente se o limite

\begin{equation}
 \nonumber \int_1^{\infty} f = \lim_{A \to \infty}\int_{1}^A f
\end{equation}

existe. O poder deste teste é demonstrado pelo Spivak ao se demonstrar a convergência para a série

\begin{equation}
 \nonumber \sum_{i=1}^{\infty} \frac{1}{n^p}
\end{equation}

Onde se $p>0$ a série converge, o que resulta na convergência de uma infinidade de séries 
(apenas variando $p$).

Até o presente momento, apenas lidamos com séries que tinham valores unicamente positivos,
o que acaba resultando que também sabemos lidar com séries que tenham todos os seus termos $\leq 0$
, já que

\begin{equation}
 \nonumber \sum_{i=1}^{\infty} a_n = -(\sum_{i=1}^{\infty} - a_n )
\end{equation}

ou seja, podemos tratar de séries que contenham termos negativos de maneira análoga as séries
positivas que vinhamos tratando.

Entretanto, diversas vezes podemos ter séries que tem tanto termos negativos como positivos, 
como exemplificado pelo polinômio de Taylor para a função $sen(x)$ com $a=0$. Para um primeiro tratamento
para estes casos, poderiamos pensar em análisar $|a_n|$.

Para isso, é interessante primeiro definir o que é uma série \textbf{absolutamente convergente}:

Uma série $\sum a_n$ é \textbf{absolutamente convergente} se a série $\sum |a_n|$ é convergente.

Apesar de que esta definição parecer ter pouca utilidade, ela acaba sendo muito poderosa, como
mostrado no teorema 5. Neste teorema, é demonstrado que se uma série $\sum a_n$ é absolutamente convergente
($\sum |a_n|$ converge), então a série $\sum a_n$ é convergente.

Ou seja, a partir de uma série convergente com termos não-negativos, pode-se obter infinitas outras séries
apenas se inserindo sinais negativos aleatoriamente. Este resultado também leva ao teorema 6, que é chamado
de teorema de Leibniz, onde, supondo uma sequência $\{a_n\}$ não-negativa e considerando que

\begin{equation}
 \nonumber \lim_{n\to\infty} a_n = 0
\end{equation}

então a série


\begin{equation}
 \nonumber \sum_{i=1}^\infty (-1)^{n+1}a_n = a_1 - a_2 + a_3 + ... 
\end{equation}

converge. 

A partir do teorema 6, é mostrado como a série

\begin{equation}
 \nonumber \sum_{i=1}^\infty (-1)^{n+1} \frac{1}{n}
\end{equation}

é convergente, apesar de não ser absolutamente convergente. Tais séries são chamadas de 
condicionalmente convergentes. Através desta série, o Spivak introduz um absurdo ao manipular
esta como se fosse uma soma finita, e não infinita. 

Além do exemplo fornecido pelo Spivak, em sala de aula também vimos outros exemplos que 
exemplificam este absurdo. Como a atribuição de que

\begin{equation}
 \nonumber -1 + 1 -1 + 1 - 1 + 1 ... = 1/2
\end{equation}

ou que 

\begin{equation}
 \nonumber \sum_{i=1}^\infty n = -1/12
\end{equation}

Para provar que isto não pode ser realizado para qualquer série, é apresentado o teorema 7,
que indica que só é possível se realizar uma manipulaçào dos termos de uma série como se esta
fosse uma soma finita, caso esta série seja absolutamente convergente. O teorema 7 diz que:

Se $\sum a_n$ converge mas não converge absolutamente, então $\forall \alpha$ existe um rearrange
de $\{a_n\}$, $\{b_n\}$, t.q.

\begin{equation}
 \nonumber \sum_{i=1}^\infty b_n = \alpha
\end{equation}

Com este teorema, mostramos que manipulando uma série que não é absolutamente convergente
podemos chegar em qualquer valor desejado, apenas realizando as manipulações ``corretas''. 
Ao unirmos o teorema 7 acima com o teorema 8, conseguimos então caracterizar quando podemos 
realizar uma manipulação de séries como se estas fossem somas finitas.

O teorema 8 diz que se $\sum a_n$ converge absolutamente, então qualquer rearrango
de $\{a_n\}$, $\{b_n\}$ também converge e 

\begin{equation}
 \nonumber \sum_{i=1}^\infty a_n = \sum_{i=1}^\infty b_n 
\end{equation}

Aqui podemos ver como é importante a caracterização de uma série absolutamente convergente,
embora em um primeiro momento ao se analisar a definição se parecia de que esta não seria tão util,
como se estivessemos perdendo informação sobre a série ao se fazer esta análise.

Entretanto, percebe-se que esta característica influência significativamente no comportamento de uma série
e de que manipulações são possíveis se realizar com esta.

Outro ponto em que esta característica se mostra fundamental é quando queremos somar ou multiplicar
séries infinitas, sendo que a relação

\begin{equation}
 \nonumber \sum_{i=1}^\infty a_n + \sum_{i=1}^\infty b_n  = \sum_{i=1}^\infty (a_n + b_n)  
\end{equation}

Só vale caso $\sum a_n$ e $\sum b_n$ sejam séries absolutamente convergentes. Outro caso em que isto
se mostra necessário é no teorema 9, onde se $\sum a_n$ e $\sum b_n$ convergem absolutamente e $\{c_n\}$
é uma sequência que contem todos os pares $a_ib_j$ então

\begin{equation}
 \nonumber \sum_{i=1}^\infty c_n = \sum_{i=1}^\infty a_n \cdot \sum_{i=1}^\infty b_n  
\end{equation}

\newpage
\section{Convergência uniforme e séries de potência}

O capitulo 24 se inicia mostrando a idéia geral do objeto de estudo deste, ao invés de 
trabalharmos com a soma de sequências infinitas, vamos lidar agora com funções definidas
pela adição de infinitas funções que dependem de $x$, como no exemplo abaixo:

\begin{equation}
	\nonumber e^x = 1 + x/1! + x^2/2! + ...
\end{equation}

Ou seja,

\begin{equation}
	\nonumber f(x) = f_1(x) + f_2(x) + ...
\end{equation}

Onde definimos uma sequência de funções $s_n$ como

\begin{equation}
	\nonumber s_n = f_1 + f_2 + ... + f_n
\end{equation}

Onde expressamos $f(x)$ como o seguinte limite:

\begin{equation}
	\nonumber f(x) = \lim_{n \to \infty} f_n(x)
\end{equation}

O spivak resume muito bem a expectividade ao se chegar neste capitulo, dada todas as coisas
que aprendemos sobre séries no capítulo passado, tudo que achavamos que seria verdade não é. (há)
E assim, ele inicia o capítulo introduzindo uma série de contra-exemplos pra entendermos quais critérios
são necessários para avaliar este tipo de função.

O primeiro exemplo envolve a soma de funções continuas para fornecer uma função não-contínua 
ao final. Ou seja, enquanto $f_1, f_2, ... f_n$ são todas funções contínuas, $\lim f_n$ não é
contínua. Em seguida ele mostra como funções diferenciaveis (!) podem ser combinadas para fornecer
uma função não-contínua.

Em seguida, é apresentado um caso ainda mais bizarro, onde uma combinações de funções não contínuas
(e extremamente erráticas - estas variam enormemente próximas a $0$) geram uma função contínua quando
avaliamos $\lim_{n \to \infty} f_n(x)$.

Para conseguirmos um critério com que faça com que estas sequências de funções se adequem a 
uma análise, precisamos definir o que é uma convergência uniforme de $f_n$ para $f$ em $A$ 
(ou dizemos que $f_n$ se aproxima uniformemente de $f$ em $A$). Esta definição se assemelha a de 
limites de uma certa maneira, onde para qualquer $\epsilon >0$ existe um $N$ t.q. para todo
$x$ em $A$:

\begin{equation}
	\nonumber se \ n > N \text{, então } |f(x) - f_n(x)| < \epsilon
\end{equation} 

A partid dessa definição, podemos então enunciar o teorema 1, que fala sobre o comportamente das 
integrais de $f_n$.

Se uma sequência de funções $\{f_n\}$ é integrável em $[a,b]$ e $\{f_n\}$ converge uniformemente para
$f$ em $[a,b]$, e se $f$ é também integrável em $[a,b]$ então

\begin{equation}
	\nonumber \int_a^b f = \lim_{n\to\infty} \int_a^b f_n
\end{equation}

Da mesma maneira que podemos tratar a integração desta sequência de funções, também podemos tratasr
de maneira similar a continuidade destas (teorema 2) caso $\{f_n\}$ converja para f em $[a,b]$.

Embora os teoremas 1 e 2 tratem com sucesso de integração e continuidade caso $\{f_n\}$ converja
para $f$ em um intervalo, a diferenciabilidade não obtém o mesmo sucesso imediatamente.
Entretanto, caso adicionemos uma hipótese, não dizendo apenas que $\{f_n\}$ converja
para $f$ em um intervalo, mas exigindo também que $\{f_n'\}$ converja uniformemente para $f'$, 
conseguimos ter uma relação. (teorema 3)

Estes 3 teoremas formam a base para o tratamento de sequências de funções, sendo que apenas nos
interessamos por funções que converjam uniformemente (sendo esta convergência definida no coralário 1)
, ou seja, tenham sua continuidade, integração e diferenciabilidade bem definidos (com todas as hipóteses que acompanham estas propriedades)
.

Agora, mesmo tendo um critério bem definido para sequências de funções ``bem comportadas'', ainda não 
sabemos avaliar quando tal sequência de fato é uniformemente convergente. Para isso, é então apresentada
o teste M de Weierstrass. 

Seja $\{f_n\}$ uma sequência de funções definida em $A$ e suponha que $\{M_n\}$ é uma sequência de números
tal que 

\begin{equation}
	\nonumber |f_n(x) | \leq M_n
\end{equation}

suponha ainda que a série

\begin{equation}
	\nonumber \sum_{n=1}^\infty M_n
\end{equation}

converge. Então, para cada $x$ em $A$ a série

\begin{equation}
	\nonumber \sum_{n=1}^\infty f_n(x)
\end{equation}

também converge absolutamente e converge uniformemente para $f(x)$ em $A$.

Com este teste, podemos agora analisar uma função que quando estavamos falando de diferenciabilidade 
e continuidade, não tinhamos ainda o ferramentário para analisar com cuidado.

No teorema 5, o Spivak mostra que a função

\begin{equation}
	\nonumber f(x) = \sum_{n=1}^\infty \frac{1}{10^n}\{10^nx\}
\end{equation}

é continua $\forall x$ porém não é diferenciável para nenhum $x$!

Com esse teste, fica muito mais simples se analisar funções bem comportadas, como funções polinomiais.

Sendo assim, já que tinhamos uma maneira de gerar séries de funções através dos polinômios de 
Taylor, faz sentido que agora nos atentemos a analisar este tipo de série.

Séries infinitas para os polinômios de Taylor podem ser descritas pela equação

\begin{equation}
	\nonumber \sum_{n=0}^\infty f_n(x) = \sum_{n=0}^\infty a_n(x-a)^n
\end{equation}

Sendo estas funções $f_n$ chamadas de séries de potência centradas em $a$. Pela simplicidade,
(e assim como no capítulo dos polinômios de Taylor), vamos apenas analisar séries de potências
centradas em $0$.

Primeiramente é importante análisar que, assim como visto no capítulo 20, nem sempre será verdade que

\begin{equation}
	\nonumber f(x) = \sum_{n=0}^\infty \frac{f^{(n)}(a)}{n!}(x-a)^n
\end{equation}

, para isso necessitamos que $\lim_{n\to\infty}R_{n,a}(x) = 0$,
porém, se uma série de potências de fato converge para uma determinada função dentro de um intervalo
$A$, podemos dizer muito sobre esta série.

Neste momento entramos no teorema 6, que diz que se uma série

\begin{equation}
	\nonumber f(x_0) = \sum_{n=0}^\infty a_n x_0^n
\end{equation}

converge, e sendo $a$ um número t.q. $0 < a < |x_0|$, então no intervalo $[-a,a]$ a série

\begin{equation}
	\nonumber f(x) = \sum_{n=0}^\infty a_n x^n
\end{equation}

converge absolutamente, resultando também que a série

\begin{equation}
	\nonumber g(x) = \sum_{n=1}^\infty n a_n x^{(n - 1)}
\end{equation}

também converge absolutamente, sendo ainda que $f$ é diferenciável e $f'(x) = g(x)$.

Agora estamos em posição de conseguir fazer várias manipulações com nossas séries de potência, como por exemplo.
supondo duas séries de potência convergentes, temos que 

\begin{equation}
	\nonumber \sum_{n=1}^\infty a_n x^n + \sum_{n=1}^\infty b_n x^n  = \sum_{n=1}^\infty (a_n + b_n) x^n 
\end{equation}

e assim como para soma de séries de potências, também podemos facilmente lidar com o produto destas.
Além disso, aplicando o teorema 6 para as funções $sen(x), cos(x)$ e  $e^x$, obtemos séries
de potência definidas e convergentes. Já para algumas outras funções, como $log(1+x)$ e $arctan(x)$, não
obtemos um comportamento tão bem ``comportado''. 

Assim como suas primeiras derivadas são bem definidas (para as funções sen, cos e exponencial),
também é apresentado como suas $k$ derivadas são bem definidas e todas as séries convergem.
Ou seja, temos em mãos uma ferramenta para aproximar as k-ésimas derivadas das funções acima
citadas.

Em suma, ao final deste capítulo conseguimos desenvolver o ferramentário para analisar os 
polinômios de Taylor como séries infinitas, onde as funções $sen,cos,e^x$  se comportam
tão bem como gostariamos, convergindo para $\forall x$, sendo diferenciável termo a termo.
Ainda assim, para algumas funções ainda não temos o ferramentário necessário para tirar 
muitas conclusões. Por exemplo a função $arctan(x)$, ainda necessita da introdução de números 
complexos para que possamos responder a questões mais profundas sobre esta série de potências.



\newpage

\end{document}
